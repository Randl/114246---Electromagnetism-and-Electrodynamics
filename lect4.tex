\subsection{Solution with Fourier transform}
$$f_k = \int \dd{x} e^{ixk} F(x) \iff \int_{-\infty}^\infty \frac{\dd{k}}{2\pi} e^{-ixk} f_k $$
$$\dd[2]{F}{x} = \int  \frac{\dd{k}}{2\pi} (-k^2) e^{-ikx}  f_k$$
Since $\laplacian{\phi} = -4\pi \rho$
$$\int (-k^2) \phi_k e^{-ikx} \dd[3]{k} = -4\pi \int \rho_k e^{-ikx} \dd[3]{k}$$
Now
$$\delta\qty(\va{k} - \va{k'}) = C\int e^{-i \va{k} x} e^{i\va{k'} x} \dd{x}$$
Thus, this is orthogonal basis and coefficients have to be equal
$$-k^2 \phi_{\va{k}} = -4\pi \rho_{\va{k}}$$
$$k^2 \phi_{\va{k}} = 4\pi \rho_{\va{k}}$$
\paragraph{Example}
$\rho_k = 0$, then
$$k^2 \phi_{\va{k}} = 0$$
i.e., either $$k^2 =0 \text{ or } \phi_{\va{k}}$$
However, $k_x, k_y, k_z > 0$, else, $e^{-ikx} = e^{|k|x} \to \infty$, which doesn't fulfills boundary conditions, thus $\phi = 0$.
\paragraph{Finite boundary conditions}
In this case, we have a Fourier series instead of transform.
\paragraph{Example}
If $\rho \neq 0$;
$$\phi (\vec{x}) = \int \frac{\rho\qty(\va{x'})}{\qty|x-x'|} \dd[3]{x}$$
i.e.
$$\phi_k = \frac{4\pi \rho_k}{k^2}$$

\subsection{Multipole expansion}
If we are far from a set of charges we can approximate them as a single charge $Q$:
$$\phi = \frac{Q}{r} = \frac{\sum_i q_i}{r} = \frac{\int \rho\qty(r') \dd[3]{r'}}{r}$$
This is monopole approximation.

Now if we denote $f\qty(\va{r'}) = \frac{1}{\qty|\va{r} -\va{r'}|}$
We can take Taylor expansion of it:
$$f\qty(\va{r'}) = f\qty(\va{r'} = 0)  + \sum_{\alpha=1}^{3} \eval{\pdv{f}{x'_\alpha}}_{\va{r'} =0} x'_\alpha + \sum_{\alpha, \beta = 1}^{3} \eval{\pdv{f}{x'_\alpha}{x'_\beta}}_{\va{r'} =0} x'_\alpha x'_\beta+ \dots$$

Then the monopole approximation is first element in the series:
$$f\qty(\va{r'}) \approx f\qty(\va{r'} = 0)$$
$$\phi_0\qty(\va{r}) = \frac{Q}{r}$$
Dipole expansion is
$$\phi_1\qty(\va{r}) = \int \dd[3]{r'} \rho(r) \pdv{r'} \frac{1}{\qty|\va{r}-\va{r'}|} \vdot \va{r'} $$
$$\pdv{r'} f = \grad{f} = \sum_{\alpha}  \pdv{f}{x_\alpha} \vu{x}_\alpha$$
i.e.,
$$\frac{1}{\qty|\va{r}-\va{r'}|} = \frac{1}{r} + \frac{\va{r} \vdot \va{r'}}{r^3} + \order{r^2}$$
meaning
$$\phi_1 = \int \dd[3]{r'} \rho(\va{r'}) \frac{\va{r} \vdot \va{r'}}{r^3}  =  \frac{\va{r}}{r^3} \underbrace{ \int \dd[3]{r'} \rho(\va{r'}) \va{r'} }_{va{P}}$$
\paragraph{Example}
A single point charge in point $\va{r'} = \va{r}_q$.
$$\phi = \frac{q}{\qty|\va{r} - \va{r}_q|} \approx \phi_1\qty(\va{r}) + \phi_2\qty(\va{r}) = \frac{q}{r} + \frac{\va{r} \vdot P}{r^3} = \frac{q}{r}+ \frac{q \va{r} \vdot \va{r}_q}{r^3}$$
\paragraph{Example}
Two point charge. Then
$$\va{P} = q_1 \va{r}_1 + q_2 \va{r}_2$$
\paragraph{Quadruple expansion}
Note that
$$\pdv{}{x'_\alpha}{x'_\beta} \eval{\frac{1}{\qty|\va{r} - \va{r'}|}}_{\va{r'}=0} = \pdv{}{x_\alpha}{x_\beta} \frac{1}{r}$$
(since $$\pdv{}{x'_\alpha}{x'_\beta} \eval{\frac{1}{\qty|\va{r} - \va{r'}|}} = - \pdv{}{x_\alpha}{x_\beta} \eval{\frac{1}{\qty|\va{r} - \va{r'}|}}$$).
Also note that
$$\sum_{\alpha, \beta} \delta_{\alpha, \beta} \pdv{}{x_\alpha}{x_\beta} \frac{1}{r} = 0$$
Now
$$\phi_2 = \sum_{\alpha, \beta}\int \dd[3]{r'} \frac{1}{2} \pdv{\frac{1}{r}}{x_\alpha}{x_\beta} x'_\alpha x'_\beta \rho(\va{r'}) = \frac{1}{2} \sum_{\alpha, \beta} \int \dd[3]{r'} \pdv{\frac{1}{r}}{x_\alpha}{x_\beta} \qty( x'_\alpha x'_\beta - \frac{1}{3} {r'}^2 \delta_{\alpha, \beta}) \rho(\va{r'}) $$
We can rewrite as
$$\phi_2  = \frac{1}{6} \pdv{\frac{1}{r}}{x_\alpha}{x_\beta} Q_{\alpha \beta}$$
where
$$Q_{\alpha \beta}= \sum_{\alpha, \beta} \int \dd[3]{r'} \rho(\va{r'}) \qty( x'_\alpha x'_\beta - \frac{1}{3} {r'}^2 \delta_{\alpha, \beta})   $$
and
$$\pdv{\frac{1}{r}}{x_\alpha}{x_\beta} = 3 \frac{x_\alpha x_\beta}{r^5} - \frac{\delta_{\alpha, \beta}}{r^3}$$
Also
$$\tr Q = \sum_{\alpha, \beta} \delta_{\alpha, \beta} Q_{\alpha, \beta} = 0$$
\paragraph{Legendre polynomial}
Define functions $P_l(x)$:
$$G(t,x) = \frac{1}{\sqrt{1-2tx +t^2}} = \sum_{l=0}^\infty P_l(x) t^l$$
It can be shown that those are orthogonal functions:
$$\int_{-1}^1 P_l(x) P_{l'}(x) \dd{x} = \frac{2}{2l+1} \delta_{ll'}$$
$$\frac{1}{\qty|\va{r} - \va{r'}|} = \frac{1}{\sqrt{r^2+{r'}^2 -2rr'\mu }} = \frac{1}{r_>} \frac{1}{\sqrt{1-2\frac{r_<}{r_>} \mu + \frac{r_<^2}{r_>^2}}} $$
where $r_>$ and $r_<$ are the bigger and smaller out of $r$ and $r'$ correspondingly, and $\mu = \cos(\va{r} \vdot \va{r'})$.

Taking $t=\frac{r_<}{r_>}$ and $x=2\mu$ we get
$$\frac{1}{\qty|\va{r} - \va{r'}|} = \frac{1}{r_>} \frac{1}{\sqrt{1-2\frac{r_<}{r_>} \mu + \frac{r_<^2}{r_>^2}}}  = \sum_{l=0}^\infty \frac{1}{r_>}\qty(\frac{r_<}{r_>})^l P_l(\mu)$$

We get
$$\phi = \int \dd[3]{r'} \rho\qty(\va{r'}) \sum_{l=0}^\infty \frac{1}{r} \qty(\frac{r'}{r})^l P_l(\mu) $$
we can rewrite, using spherical harmonics
$$P_l(\mu') = \sum_{n=-l}^l \frac{4\pi}{2l+1} Y_{lm}\qty(\vu{r})Y^*_{lm}\qty(\vu{r'})$$