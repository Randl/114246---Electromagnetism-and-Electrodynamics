
where
$$(\va{k} \vdot \grad)\va{U} = \sum_{\alpha, \beta } k_\alpha  \pdv{U_\beta}{x_\alpha} \vu{x}_\beta$$
We can rewrite as

$$(\va{k} \vdot \grad)\va{U} = \sum_{\alpha } \qty(k_\alpha \vu{x}_\alpha) \vdot \sum_{\alpha, \beta } \qty(\pdv{U_\beta}{x_\alpha} \vu{x}_\beta \vu{x}_\alpha)$$
Denote matrix of derivatives $\pdv{U_\beta}{x_\alpha} \vu{x}_\beta \vu{x}_\alpha$ as $\grad{\va{U}}$, i.e., dyadic or outer product of $\grad$ and $\va{U}$.
Then
$$\oint \dd{\va{r'}} \cross \va{U}(\va{r}) = \int \grad{\va{U}} \vdot \dd{\va{s}} - \int \dd{s} \div{\va{U}}$$
In our case $\va{U} =\frac{\va{r}-\va{r'}}{\abs{\va{r}-\va{r'}}^3}$ and since $\va{r} \neq \va{r'}$, $\div{\va{U}} = 0$.
Finally, we got
$$\va{B}(\va{r}) = \frac{I}{c} \int \grad_{r'}{\frac{\va{r}-\va{r'}}{\abs{\va{r}-\va{r'}}^3}} \vdot \dd{\va{s'}} = -\frac{I}{c} \int \grad_r{\frac{\va{r}-\va{r'}}{\abs{\va{r}-\va{r'}}^3}} \vdot \dd{\va{s'}} = -\frac{I}{c} \grad_r{\int \frac{\va{r}-\va{r'}}{\abs{\va{r}-\va{r'}}^3}} \vdot \dd{\va{s'}}$$
That means
$$\phi_m(\va{r}) = \frac{I}{c} \grad_r{\int \frac{\va{r}-\va{r'}}{\abs{\va{r}-\va{r'}}^3}} \vdot \dd{\va{s'}}$$
such that
$$\va{B}= -\grad \phi_m(\va{r})$$

\paragraph{Geometrical meaning of $\phi_m$}
Suppose $\va{r}=0$. Then

$$\phi_m(\va{r}) = \frac{I}{c} \grad_r{\int -\frac{\va{r'}}{{r'}^3}} \vdot \dd{\va{s'}}$$
Note that $\frac{\va{r'}\vdot \dd{\va{s'}}}{{r'}^3} $ is solid angle $\Omega$ of current loop, i.e.
$$\phi_m = \frac{I}{c} \Omega$$

\subsection{Multipole expansion of magnets}
First of all for $\va{E}$ monopole is sum of charges, however there is no such thing as magnetic monopole.
\paragraph{Exercise}
Suppose there exists magnetic monopole $\va{B}= \frac{Q_B \vu{r}}{r^2}$. Solve the problem of movement
$$m\ddot{r} = \va{F}_B = \frac{q}{c} \va{v} \cross \va{B}$$

Write down the scalar potential
$$\phi_m = \frac{I}{c} \int \dd{s'} \cdot \frac{\va{r} - \va{r'}}{\abs{\va{r} - \va{r'}}^3}$$
\paragraph{First order approximation}
$r' \approx 0$:
$$\phi_m \approx \frac{I}{c} \int  \dd{s'} \vdot \frac{\va{r}}{r^3} = \frac{I}{c} \frac{\va{r}}{r^3}  \vdot  \int  \dd{s'} = \va{M} \vdot \frac{\va{r}}{r^3}$$
where
$$\va{M}= \frac{I}{c} \va{S}$$
is magnetic moment.

Alternatively we could use vector potential:
$$\va{A} = \frac{1}{c} \int \dd[3]{r'} \frac{\va{j}(\va{r'})}{\abs{\va{r}-\va{r'}}}$$
Taking Taylor series of $\frac{1}{\abs{\va{r}-\va{r'}}}$ around 0, we get
$$\va{A} \approx \frac{1}{cr} \int \va{j}(\va{r'}) \dd[3]{r'} +  \frac{\va{r}}{cr^3} \vdot \int \va{r'} \va{j}(\va{r'}) \dd[3]{r'}$$
($\frac{1}{\abs{\va{r}-\va{r'}}} \approx \frac{1}{r} + \frac{\va{r}\vdot \va{r'}}{r^3}$, thus we have a matrix under integral).
Note since
$$\pdv{x'_\beta}{x'_\alpha} = \delta_{\alpha\beta}$$
is identity matrix, we can rewrite
$$\int \va{j} \dd[3]{r'} =\int j_\alpha \cdot \pdv{x'_\beta}{x'_\alpha} \vu{x}_\beta \dd[3]{r'} = \int \pdv{x'_\alpha}\qty(j_\alpha x'_\beta \vu{x}_\beta) - x'_\beta \vu{x}_\beta\underbrace{ \pdv{j_\alpha}{x'_\alpha}}_{\div{\va{j}} = 0}\dd[3]{r'}$$
In matrix form:
$$\int \va{j} \dd[3]{r'} =\int \va{j} \vdot \pdv{x'_\beta}{x'_\alpha} \dd[3]{r'} = \int \div(\va{j}\va{x'})  - \qty(\div{\va{j}} )\va{r'} \dd[3]{r'}$$
Now, since there are no currents
$$\int \pdv{x'_\alpha}\qty(j_\alpha x'_\beta \vu{x}_\beta) \dd[3]{r'} = \int \dd{s'_\alpha} j_\alpha x'_\beta \vu{x}_\beta = 0$$
$$ \int \div(\va{j}\va{x'})  =  \int \dd{s'_\alpha} \va{j}\va{x'}  = 0$$
Thus
$$\int \va{j} \dd[3]{r'} =0$$


Now for second-order part, take one element of it:
$$ \eval{\int \va{r'} \va{j}(\va{r'}) \dd[3]{r'}}_{\alpha} = \int j_\beta \vu{x}'_\beta x'_\alpha \dd[3]{r'} = \frac{1}{2} \int j_\beta x'_\beta  x'_\alpha \dd[3]{r'} + \frac{1}{2} \int x'_\beta \vu{x}'_\beta j_\alpha \dd[3]{r'} + \frac{1}{2} \int j_b \vu{x}'_\beta x'_\alpha \dd[3]{r} - \frac{1}{2} \int x'_\beta \vu{x}'_\beta j_\alpha \dd[3]{r} $$