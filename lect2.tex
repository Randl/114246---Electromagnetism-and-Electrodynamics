Suppose we have infinite grounded plane and a point charge in distance $a$ from it. The bound condition is $\phi(x=0,y,z) = 0$ And the potential of point charge is
$\laplacian{\Phi} = -4\pi q \delta \qty(\va{r} - \va{r}_0)$

The way to solve this kind of problems is to add imaginary charge such that we get zero potential on bound (e.g., symmetrically with opposite charge).

If we have two such charges:
$$\Phi = \frac{q}{(x-a)^2+y^2+z^2} -  \frac{q}{(x+a)^2+y^2+z^2}$$
This fulfills bound conditions:
 $$\phi(x=0,y,z) = \frac{q}{a^2+y^2+z^2} -  \frac{q}{a^2+y^2+z^2}= 0$$
 
 Denote density of charge on plane as $\sigma$, then $\dd{q} = \sigma \dd{s}$. If this case force between plane and charge is $\abs{\frac{\sigma q \dd{s}}{R^2}}$
 
 How do we find $\sigma$? From Gauss law
 $$\oint \vec{E} \cdot \dd{s} = 4\pi \int\limits_V \rho \dd[3]{\vb{r}}$$
 $$-|E_n|\dd{s} = 4\pi \sigma \dd{s}$$
 $$4\pi \sigma = E_n$$
 
 \subsection{Green function}
 Suppose we have a unit charge in point $\va{r'}$. Then potential is $\Phi_{\va{r'}}$ such that $\Delta \Phi = -4\pi \delta (\va{r} - \va{r'}) $.
 $$G\qty(\va{r}, \va{r'}) = \Phi_{\va{r'}}$$
 Thus
 $$ \laplacian{G\qty(\va{r}, \va{r'})} = -4\pi \delta \qty(\va{r} - \va{r'}) $$
 Then we can write $G$ as
 $$G\qty(\va{r}, \va{r'} ) = \frac{1}{\abs{r-r'}} + F\qty(\va{r}, \va{r'} )$$ 
 And $F$ guaranties bound conditions, while $ \laplacian{F} = 0$.
 
 \paragraph{Bound conditions}
 \begin{enumerate}
 	\item $G\qty(\va{r}, \va{r'}) = 0$.for bound $S$.
 	\item $ \pdv{\vu{n}}G(\va{r}, \va{r'}) = -\frac{4\pi}{S} $. Then number comes from Gauss' law
 \end{enumerate}

\paragraph{Green theorem}
$$\int\limits_V \left( \phi \grad{\psi} - \psi  \grad{\phi} \right) \dd[3]{\vb{r}} = \oint\limits_S \left[ \phi \pdv{\psi}{ \vu{n}} - \psi \pdv{\phi}{ \vu{n}} \right]\dd{s} $$

Since
$$\int\limits_V \dd[3]{\vb{r}} \div(\phi\grad{\psi}) = \oint \phi \grad{\psi} \cdot \dd{\va{s}}= \oint \phi \pdv{\psi}{ \vu{n}} \dd{s}$$
we get

$$\int\limits_V \dd[3]{\vb{r}} \left( \phi  \laplacian{\psi} - \psi \laplacian{ \phi} \right)   = \int\limits_V \dd[3]{\vb{r}} \div(\phi \grad{\psi}) - \grad{\phi} \vdot \grad{\psi} - \div\left(\phi \grad{\psi}\right) + \grad{\psi} \vdot \grad{\phi }= \oint\limits_S \left[ \phi \pdv{\psi}{\vu{n}} - \psi \pdv{\phi}{\vu{n}} \right]\dd{s}$$ %TODO


Now for $\psi\qty(\va{r}) = G\qty(\va{r}, \va{r'})$
$$\int\limits_V \dd[3]{\vb{r}} \left( - \phi \cdot 4\pi \delta \qty(\va{r}-\va{r'}) - G \laplacian{\phi} \right)  = \oint\limits_S\dd{s} \left[ \phi \pdv{ G}{ \vu{n}} - G \pdv{ \phi}{ \vu{n}}\right] $$

Suppose we want $\laplacian{\phi} = 4\pi \rho(\va{r})$ for some $\rho$ and Dirichlet bound conditions on same surface $S$:
$$\int\limits_V \dd[3]{\vb{r}}  \left( -4\pi\phi\qty(\va{r'}) - G \laplacian{\phi} \right) = \oint\limits_S\dd{s} \left[ \phi \pdv{G}{\vu{n}} - G \pdv{\phi}{\vu{n}} \right] $$
Substituting $G=0$ on the bound surface
$$ -4\pi\phi\qty(\va{r'}) + \int\limits_V \dd[3]{\vb{r}}  G \cdot 4\pi \rho\qty(\va{r})   = \oint\limits_S \dd{s}  \phi\pdv{G}{\vu{n}}  $$
i.e.
$$ \phi\qty(\va{r'}) = \int\limits_V    G \qty(\va{r}, \va{r'})\rho\qty(\va{r}) \dd[3]{\vb{r}} - \oint\limits_S  \frac{\phi }{4\pi}\cdot \pdv{G}{\vu{n}} \dd{s} $$

\paragraph{Neumann bound conditions}
$$-4\pi \phi\qty(\va{r'}) + 4\pi \int \rho\qty(\va{r}) G\qty(\va{r}, \va{r'}) \dd[3]{\vb{r}} = \oint \dd{s} \left[ \phi \left( -\frac{4\pi}{S}\right) -  G\pdv{\phi}{\vu{n}} \right] $$
However
$$ \oint  \phi \left( -\frac{4\pi}{S}\right)  \dd{s} = -\frac{4\pi}{S} \oint \phi \dd{s} = -4\pi \langle  \phi \rangle$$
$$\phi\qty(\va{r'}) =  \int \rho\qty(\va{r}) G\qty(\va{r}, \va{r'}) \dd[3]{\vb{r}} +  \frac{1 }{4\pi}\oint   G\pdv{\phi}{\vu{n}} \dd{s} + \langle  \phi \rangle$$
\paragraph{Example}
$$G\qty(\va{r}, \va{r'}) = \frac{1}{\abs{\va{r} - \va{r'}}}$$
With bound conditions
$$0 = G\qty(\va{r}\to \infty, \va{r'})$$
Thus
$$\phi\qty(\va{r}) = \int \rho \qty(\va{r}) \frac{1}{\abs{\va{r} - \va{r'}}} \dd[3]{\vb{r}}$$
\paragraph{Example}
Suppose we have an infinite plane with potential $\phi(x=0,y,z)=\phi_S(y,z)$ and $\rho=0$ in one side of space ($x>0$). We are searching for $\phi(x>0,y,z)$.

Define Green function as a solution of previous problem, with point charge and grounded plane:
$$G\qty(\va{r}, \va{r'}) = \frac{1}{\qty\Big[(x-x')^2+(y-y')^2+(z-z')^2]^{\frac{1}{2}}}-\frac{1}{\qty\Big[(x+x')^2+(y-y')^2+(z-z')^2]^{\frac{1}{2}}}$$
