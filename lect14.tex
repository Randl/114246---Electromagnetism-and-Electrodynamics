% TODO
\paragraph{4-vector of energy and momentum}
$$P^i = \qty(\frac{mc}{\sqrt{1-\frac{v^2}{c^2}}} +\frac{q}{c}\phi, \va{P} + \frac{q}{c}\va{A})$$
$$A_{i}\eta^{ij} = A^j$$
$$A_i = \eta_{ij}A^j$$
\paragraph{Invariants of electromagnetic field}
In case of tensor
$$F_k^i = \eta_{kj}F^{ij}$$
$$F_{kl} = \eta_{li}\eta_{kj}F^{ij}$$
$$F_{ij}F^{ij} = B^2-E^2$$
and
$$\epsilon^{ijkl}F_{ij}F^{kl} = \va{E} \vdot \va{B}$$
are invariant, where $\epsilon^{iklm}$ is Levi-Civita tensor.

That means that if $\va{E} \perp \va{B}$ or $\abs{\va{E}} = \abs{\va{B}}$

Also if $\va{E} \vdot \va{B} = 0$ there exists frame of reference where one of them is zero (which one depends on $E^2-B^2$). If one of invariants is non-zero, there always exists frame of reference where they are parallel.

$$\epsilon^{ijkl}F_{ij}F_{kl} = 4 \pdv{x^i} \qty(\epsilon^{ijkl}A_k \pdv{x_l} A_m)$$
which is $4D$ divergence. 

\paragraph{Maxwell equations}
$$\dv{\va{P}}{t} = q\va{E} + \frac{q}{c} \va{v} \cross \va{B}$$
where
$$\va{E} = -\grad{\phi} - \frac{1}{c} \pdv{\va{A}}{t}$$ 
and
$$\va{B} = \curl{\va{A}}$$
Thus
$$\curl{E} = -\frac{1}{c} \pdv{t} \curl{\va{A}} = -\frac{1}{c} \pdv{\va{B}}{t}$$
and
$$\div{\va{B}} = \grad(\curl{A}) = 0$$
i.e., first pair of Maxwell equations is determined exclusively by movement equations and doesn't require fields at all.

For second pair of Maxwell equations lets look at $\mathcal{L}$ of $A^i$ itself:
$$S = S_m+S_{mf}+S_f$$
where $S_m$ is action of particles, $S_{mf}$ is action of particle-field interactions and $S_f$ is filed action.
$$S_m = \sum_{particles} mc \int_a^b \dd{s}$$
$$S_{mf} = -\sum \frac{q}{c} \int A_k \dd{x^k}$$
$$\sum q A_k \dd{x^k} = \rho \dd[3]{x} A_k \dv{x^k}{t} \dd{t} = \rho \dv{x^k}{t} A_k \dd[3]{x} \dd{t} = j^k A_k \dd[3]{x} \dd{t} = j^k A_k \frac{\dd{\Omega}}{c}$$
where $\dd{\Omega} = \dd[4]{x}$, i.e.
$$S_{mf} = -\frac{1}{c^2} \int j^k A_k \dd{\Omega}$$
\subsection{Noether's theorem: connection between gauge invariance and charge conservation}
% Landau Field theory paragraph 27 and around
Note that $A_i \to A_i - \pdv{f}{x^i}$ doesn't change the fields:
$$\begin{cases}
\va{E} = -\frac{1}{c} \pdv{\va{A}}{t} - \grad{\phi}\\
\va{B}= \curl{\va{A}}
\end{cases}$$
Substituting
$$\begin{cases}
\tilde{\phi} = \phi - \frac{1}{c}\pdv{f}{t}\\
\va{\tilde{A}} = \va{A} + \grad{f}
\end{cases}$$
$$\curl{\va{\tilde{A}}} =\curl(\va{A} + \grad{\phi}) = \curl{\va{A}}$$
$$-\frac{1}{c} \pdv{\va{\tilde{A}}}{t} - \grad{\tilde{\phi}} = -\frac{1}{c} \pdv{t} \qty(\va{A} + \grad{f}) - \grad(\phi - \frac{1}{c} \pdv{f}{t}) = -\frac{1}{c} \pdv{\va{A}}{t} - \grad{\phi}$$

$$\tilde{S}_{mf} = -\frac{1}{c^2} \int j^i \tilde{A}_i \dd{\Omega} = -\frac{1}{c^2}\int j^i A_i \dd{\Omega} - \frac{1}{c^2}\int j^i \pdv{f}{x^i} \dd{\Omega} = S_{mf} - \frac{1}{c^2} \int \qty[\pdv{x^i} (fj^i) - f\pdv{j^i}{x^i}] \dd{\Omega}$$
From Gauss, first term of is 0, and thus the second has to be 0 too, resulting in charge conservation (or the other way around, from charge conservation we get gauge invariance).
