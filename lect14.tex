% TODO
\paragraph{4-vector of energy and momentum}
$$P^i = \qty(\frac{mc}{\sqrt{1-\frac{v^2}{c^2}}} +\frac{q}{c}\phi, \va{P} + \frac{q}{c}\va{A})$$
$$A_{i}\eta^{ij} = A^j$$
$$A_i = \eta_{ij}A^j$$
\paragraph{Invariants of electromagnetic field}
In case of tensor
$$F_k^i = \eta_{kj}F^{ij}$$
$$F_{kl} = \eta_{li}\eta_{kj}F^{ij}$$
$$F_{ij}F^{ij} = B^2-E^2$$
and
$$\epsilon^{ijkl}F_{ij}F^{kl} = \va{E} \vdot \va{B}$$
are invariant, where $\epsilon^{iklm}$ is Levi-Civita tensor.

That means that if $\va{E} \perp \va{B}$ or $\abs{\va{E}} = \abs{\va{B}}$

Also if $\va{E} \vdot \va{B} = 0$ there exists frame of reference where one of them is zero (which one depends on $E^2-B^2$). If one of invariants is non-zero, there always exists frame of reference where they are parallel.

$$\epsilon^{ijkl}F_{ij}F_{kl} = 4 \pdv{x^i} \qty(\epsilon^{ijkl}A_k \pdv{x_l} A_m)$$
which is $4D$ divergence. 

\paragraph{Maxwell equations}
$$\dv{\va{P}}{t} = q\va{E} + \frac{q}{c} \va{v} \cross \va{B}$$
where
$$\va{E} = -\grad{\phi} - \frac{1}{c} \pdv{\va{A}}{t}$$ 
and
$$\va{B} = \curl{\va{A}}$$
Thus
$$\curl{E} = -\frac{1}{c} \pdv{t} \curl{\va{A}} = -\frac{1}{c} \pdv{\va{B}}{t}$$
and
$$\div{\va{B}} = \grad(\curl{A}) = 0$$
i.e., first pair of Maxwell equations is determined exclusively by movement equations and doesn't require fields at all.

For second pair of Maxwell equations lets look at $\mathcal{L}$ of $A^i$ itself:
$$S = S_m+S_{mf}+S_f$$
where $S_m$ is action of particles, $S_{mf}$ is action of particle-field interactions and $S_f$ is filed action.
$$S_m = \sum_{particles} mc \int_a^b \dd{s}$$
$$S_{mf} = -\sum \frac{q}{c} \int A_k \dd{x^k}$$
$$\sum q A_k \dd{x^k} = \rho \dd[3]{x} A_k \dv{x^k}{t} \dd{t} = \rho \dv{x^k}{t} A_k \dd[3]{x} \dd{t} = j^k A_k \dd[3]{x} \dd{t} = j^k A_k \frac{\dd{\Omega}}{c}$$
where $\dd{\Omega} = \dd[4]{x}$, i.e.
$$S_{mf} = -\frac{1}{c^2} \int j^k A_k \dd{\Omega}$$
\subsection{Noether's theorem: connection between gauge invariance and charge conservation}
% Landau Field theory paragraph 27 and around
Note that $A_i \to A_i - \pdv{f}{x^i}$ doesn't change the fields:
$$\begin{cases}
\va{E} = -\frac{1}{c} \pdv{\va{A}}{t} - \grad{\phi}\\
\va{B}= \curl{\va{A}}
\end{cases}$$
Substituting
$$\begin{cases}
\tilde{\phi} = \phi - \frac{1}{c}\pdv{f}{t}\\
\va{\tilde{A}} = \va{A} + \grad{f}
\end{cases}$$
$$\curl{\va{\tilde{A}}} =\curl(\va{A} + \grad{\phi}) = \curl{\va{A}}$$
$$-\frac{1}{c} \pdv{\va{\tilde{A}}}{t} - \grad{\tilde{\phi}} = -\frac{1}{c} \pdv{t} \qty(\va{A} + \grad{f}) - \grad(\phi - \frac{1}{c} \pdv{f}{t}) = -\frac{1}{c} \pdv{\va{A}}{t} - \grad{\phi}$$

$$\tilde{S}_{mf} = -\frac{1}{c^2} \int j^i \tilde{A}_i \dd{\Omega} = -\frac{1}{c^2}\int j^i A_i \dd{\Omega} - \frac{1}{c^2}\int j^i \pdv{f}{x^i} \dd{\Omega} = S_{mf} - \frac{1}{c^2} \int \qty[\pdv{x^i} (fj^i) - f\pdv{j^i}{x^i}] \dd{\Omega}$$
From Gauss, first term of is 0, and thus the second has to be 0 too, resulting in charge conservation (or the other way around, from charge conservation we get gauge invariance).
\paragraph{Action of field $S_f$}
\begin{enumerate}
	\item Superposition princples dictates quadratic form in fields $\va{B}$ and $\va{E}$
	\item Action depends on fields and not on potential
	\item $\mathcal{L}$ has to be scalar, i.e. invariant under Lorentz transformation.
\end{enumerate}
Thus the action has of form $a\int F_{ik}F^{ik} \dd{\Omega}$. Since $F_{ik}F^{ik}=2(B^2-E^2)$, and $\va{E}$ contains $\pdv{\va{A}}{t}$, $a$ is negative so that action has minimum.

Thus the total action is
$$S = -\sum_{\text{particles}} \int mc \dd{s} - \frac{1}{c^2}A_i j^i \dd{\Omega} - \frac{1}{16\pi c} \int F_{ik}F^{ik} \dd{\Omega}$$
% Landau paragraph 30
Equating functional differential of $S$ to $0$:
$$0 = \delta S = -\frac{1}{c} \int \qty[\frac{1}{c} j^i \delta A_i + \frac{1}{8\pi} F^{ik}F_{ik}] \dd{\Omega}$$

Substituting
$$\delta F_{ik} = \delta \qty[\pdv{A_n}{x^i} - \pdv{A_i}{x^k}] = \pdv{\delta A_k}{x^i}-\pdv{\delta A_i}{x^k}$$
and
$$F^{ik} \pdv{\delta A_k}{x^i} =F^{ki}  \pdv{\delta A_i}{x^k} = - F^{ik} c $$
we get
$$F^{ik} \delta F_{ik} = -2 F^{ik}\pdv{\delta A_i}{x^k}  $$
and thus
$$\delta S = -\frac{1}{c} \int \qty[\frac{1}{c} j^i \delta A_i - \frac{1}{4\pi} F^{ik}\pdv{\delta A_i}{x^k} ] \dd{\Omega}$$
and since
$$F^{ik}\pdv{\delta A_i}{x^k} = \pdv{\qty(F^{ik}\delta A_i)}{x^k} - \pdv{F^{ik}}{x^k} \delta A_i$$
applying Gauss law

$$\delta S = -\frac{1}{c} \int \qty(\frac{1}{c} j^i  + \frac{1}{\pi}\pdv{F^{ik}}{x^k})\delta A_i \dd{\Omega}- \frac{1}{4\pi c} \int F^{ik} \delta A_i \dd{S_k}$$
Since surface integral is zero, we get

$$\int \qty(\frac{1}{c} j^i  + \frac{1}{\pi}\pdv{F^{ik}}{x^k})\delta A_i \dd{\Omega} = 0$$
$$ \frac{1}{\pi}\pdv{F^{ik}}{x^k} = \frac{4\pi}{c} j^i  $$


For $i=0$:
$$\pdv{F^{01}}{x}+\pdv{F^{02}}{y}+\pdv{F^{03}}{z} = -\frac{4\pi }{c} j^0$$
Since $j^0 = c\rho$, we get
$$\div{\va{E}} = 4\pi \rho$$
For $i=1$:
$$\pdv{F^{12}}{y}+\pdv{F^{13}}{z}+\frac{1}{c}\pdv{F^{10}}{t} = -\frac{4\pi }{c} j^1$$
$$\pdv{B_z}{y} - \pdv{B_y}{z} - \frac{1}{c} \pdv{E_x}{t} = \frac{4\pi}{c} j_x$$
i.e.,
$$\curl{\va{B}} = \frac{4\pi}{c} \va{j} + \frac{1}{c} \pdv{\va{E}}{t}$$
\paragraph{Energy density and energy flux}
$$\begin{cases}
\curl{\va{E}} = - \frac{1}{c} \pdv{\va{B}}{t}\\
\curl{\va{B}} =  \frac{1}{c} \pdv{\va{B}}{t} + \frac{4\pi}{c} \va{j}\\
\div{\va{B}}=0\\
\div{\va{E}} = 4\pi \rho
\end{cases}$$
$$\frac{1}{c} \va{E} \vdot \pdv{\va{E}}{t} + \frac{1}{c} \va{B}\vdot \pdv{\va{B}}{t} = \va{E} \vdot \qty[\curl{\va{B}} - \frac{4\pi}{c} \va{j}] - \va{B} \vdot \div{\va{E}} $$
$$\frac{1}{2c} \pdv{t} \qty( E^2+B^2) = - \frac{4\pi}{c} \va{j} \vdot \va{E}  - \div(\va{E} \cross \va{B})$$
$$ \pdv{t} \frac{E^2+B^2}{8\pi} = - \va{j} \vdot \va{E}  - \div(\va{S})$$
where $\va{S} = \frac{c}{4\pi} \va{E} \cross \va{B}$ is Poynting vector.

Physical meaning is that by integrating over the whole space
$$\dv{t} \int \frac{E^2+B^2}{8\pi} \dd[3]{r}= - \int \va{j} \vdot \va{E}] \dd[3]{r} - \underbrace{\oint \va{S} \vdot \dd{\va{a}}}_{0 \text{ for bounds in } \infty}$$
$$ \int \va{j} \vdot \va{E}] \dd[3]{r} = \sum q \va{v} \vdot \va{E} = \sum \va{v} \vdot \qty(m \dv{\va{v}}{t}) = \dv{t} (\sum \frac{1}{2} m v^2)$$
i.e., denoting kinetic energy by $K$,
$$\dv{t} \qty[\frac{E^2+B^2}{8\pi}  + K] = 0$$