\paragraph{Action of field $S_f$}
\begin{enumerate}
	\item Superposition princples dictates quadratic form in fields $\va{B}$ and $\va{E}$
	\item Action depends on fields and not on potential
	\item $\mathcal{L}$ has to be scalar, i.e. invariant under Lorentz transformation.
\end{enumerate}
Thus the action has of form $a\int F_{ik}F^{ik} \dd{\Omega}$ (second invariant is full derivative). Since $F_{ik}F^{ik}=2(B^2-E^2)$, and $\va{E}$ contains $\pdv{\va{A}}{t}$, $a$ is negative so that action has minimum.

Thus the total action is
$$S = -\sum_{\text{particles}} \int mc \dd{s} - \frac{1}{c^2}A_i j^i \dd{\Omega} - \frac{1}{16\pi c} \int F_{ik}F^{ik} \dd{\Omega}$$
% Landau paragraph 30
Equating functional differential of $S$ to $0$:
$$0 = \delta S = -\frac{1}{c} \int \qty[\frac{1}{c} j^i \delta A_i + \frac{1}{8\pi} F^{ik}\delta F_{ik}] \dd{\Omega}$$

Substituting
$$\delta F_{ik} = \delta \qty[\pdv{A_k}{x^i} - \pdv{A_i}{x^k}] = \pdv{\delta A_k}{x^i}-\pdv{\delta A_i}{x^k}$$
and
$$F^{ik} \pdv{\delta A_k}{x^i} =F^{ki}  \pdv{\delta A_i}{x^k} = - F^{ik} \pdv{\delta A_i}{x^k} $$
we get
$$F^{ik} \delta F_{ik} = -2 F^{ik}\pdv{\delta A_i}{x^k}  $$
and thus
$$\delta S = -\frac{1}{c} \int \qty[\frac{1}{c} j^i \delta A_i - \frac{1}{4\pi} F^{ik}\pdv{\delta A_i}{x^k} ] \dd{\Omega}$$
and since
$$F^{ik}\pdv{\delta A_i}{x^k} = \pdv{\qty(F^{ik}\delta A_i)}{x^k} - \pdv{F^{ik}}{x^k} \delta A_i$$
applying Gauss law

$$\delta S = -\frac{1}{c} \int \qty(\frac{1}{c} j^i  + \frac{1}{\pi}\pdv{F^{ik}}{x^k})\delta A_i \dd{\Omega}- \frac{1}{4\pi c} \int F^{ik} \delta A_i \dd{S_k}$$
Since surface integral is zero, we get

$$\int \qty(\frac{1}{c} j^i  + \frac{1}{\pi}\pdv{F^{ik}}{x^k})\delta A_i \dd{\Omega} = 0$$
$$ \frac{1}{\pi}\pdv{F^{ik}}{x^k} = \frac{4\pi}{c} j^i  $$


For $i=0$:
$$\pdv{F^{01}}{x}+\pdv{F^{02}}{y}+\pdv{F^{03}}{z} = \frac{4\pi }{c} j^0$$
Since $j^0 = c\rho$, we get
$$\div{\va{E}} = 4\pi \rho$$
For $i=1$:
$$\pdv{F^{12}}{y}+\pdv{F^{13}}{z}+\frac{1}{c}\pdv{F^{10}}{t} = -\frac{4\pi }{c} j^1$$
$$\pdv{B_z}{y} - \pdv{B_y}{z} - \frac{1}{c} \pdv{E_x}{t} = \frac{4\pi}{c} j_x$$
i.e.,
$$\curl{\va{B}} = \frac{4\pi}{c} \va{j} + \frac{1}{c} \pdv{\va{E}}{t}$$
\paragraph{\textcolor{red}{What if $a$ was positive?}}

$$\begin{cases}
\curl{\va{E}} = - \frac{1}{c} \pdv{\va{B}}{t}\\
\curl{\va{B}} =  -\frac{1}{c} \pdv{\va{B}}{t} + \frac{4\pi}{c} \va{j}\\
\div{\va{B}}=0\\
\div{\va{E}} = -4\pi \rho
\end{cases}$$
Same charges start to attract.

In addition, we get inversion of Lenz law, i.e. induced current increases the magnetic field and magnetic field diverges.
\paragraph{Energy density and energy flux}
$$\begin{cases}
\curl{\va{E}} = - \frac{1}{c} \pdv{\va{B}}{t}\\
\curl{\va{B}} =  \frac{1}{c} \pdv{\va{B}}{t} + \frac{4\pi}{c} \va{j}\\
\div{\va{B}}=0\\
\div{\va{E}} = 4\pi \rho
\end{cases}$$
$$\frac{1}{c} \va{E} \vdot \pdv{\va{E}}{t} + \frac{1}{c} \va{B}\vdot \pdv{\va{B}}{t} = \va{E} \vdot \qty[\curl{\va{B}} - \frac{4\pi}{c} \va{j}] - \va{B} \vdot \div{\va{E}} $$
$$\frac{1}{2c} \pdv{t} \qty( E^2+B^2) = - \frac{4\pi}{c} \va{j} \vdot \va{E}  - \div(\va{E} \cross \va{B})$$
$$ \pdv{t} \frac{E^2+B^2}{8\pi} = - \va{j} \vdot \va{E}  - \div(\va{S})$$
where $\va{S} = \frac{c}{4\pi} \va{E} \cross \va{B}$ is Poynting vector.

Physical meaning is that by integrating over the whole space
$$\dv{t} \int \frac{E^2+B^2}{8\pi} \dd[3]{r}= - \int \va{j} \vdot \va{E}] \dd[3]{r} - \underbrace{\oint \va{S} \vdot \dd{\va{a}}}_{0 \text{ for bounds in } \infty}$$
$$ \int \va{j} \vdot \va{E}] \dd[3]{r} = \sum q \va{v} \vdot \va{E} = \sum \va{v} \vdot \qty(m \dv{\va{v}}{t}) = \dv{t} (\sum \frac{1}{2} m v^2)$$
i.e., denoting kinetic energy by $K$,
$$\dv{t} \qty[\frac{E^2+B^2}{8\pi}  + K] = 0$$

\paragraph{Maxwell equations in matter}
$$\begin{cases}
\curl{\va{E}} = - \frac{1}{c} \pdv{\va{B}}{t}\\
\div{\va{B}}=0\\
\curl{\va{H}} =  \frac{1}{c} \pdv{\va{D}}{t} + \frac{4\pi}{c} \va{j}\\
\div{\va{D}} = 4\pi \rho
\end{cases}$$

Now
\begin{align*}
\curl{\va{B}} = \frac{4\pi}{c} \qty(\va{j}_f+\va{j}_b) +\frac{1}{c} \pdv{t} \qty(\va{D}-\va{P}) =\frac{4\pi}{c} \va{j}_f+\frac{4\pi}{c} \va{j}_b +\frac{1}{c} \pdv{\va{D}}{t} -\pdv{\va{P}}{t} 
\end{align*}
But
$$4\pi\va{j}_b = c \curl{\va{M}} - \pdv{\va{P}}{t} $$

\begin{align*}
\curl{\va{B}} =4\pi \curl{\va{B}} +\frac{4\pi}{c} \va{j}_f +\frac{1}{c} \pdv{\va{D}}{t} -\pdv{\va{P}}{t} 
\end{align*}