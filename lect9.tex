$$\vu{x'}_\beta \pdv{x'_\gamma} \qty(x'_\alpha x'_\beta j_\gamma ) = j_\alpha x'_\beta \vu{x'}_\beta + j_\beta x'_\alpha \vu{x'}_\beta + x'_\alpha x'_\beta \vu{x'}_\beta \underbrace{\pdv{j_\gamma}{x'_\gamma}}_{0} $$
$$\div{\va{j}\va{r}\va{r}} = \va{r}\va{j}\cdot \grad{\va{r}}) + (\va{j}\cdot \grad{\va{r}})\va{r} + \underbrace{\div{\va{j}}\va{r}}_{0}$$
Thus
$$ \frac{1}{2} \int j_\beta \vu{x'}_\beta  x'_\alpha \dd[3]{r'} + \frac{1}{2} \int x'_\beta \vu{x}'_\beta j_\alpha \dd[3]{r'} =\frac{1}{2} \int \vu{x'}_\beta \pdv{x'_\gamma} \qty(x'_\alpha x'_\beta j_\gamma )  \dd[3]{r'} =  \frac{\vu{x'}_\beta}{2} \oint  x'_\alpha x'_\beta \va{j} \vdot \dd{\va{s}}$$
$$ frac{1}{2}\int \va{r'} \va{j}\qty(\va{r'}) -\va{j}\qty(\va{r'})\va{r'}  \dd[3]{r'} = frac{1}{2}\int \div{\va{j}\va{r}\va{r}} \dd[3]{r'} = frac{1}{2} \oint \va{j}\va{r}\va{r} \dd[3]{r'}$$
If there are no currents inside, the integral is 0.

From second part we can acquire (from bac-cab identity):
$$A_2  = \frac{1}{2cr^3} \int \va{r} \vdot \qty[\va{r'} \va{j}\qty(\va{r'}) +\va{j}\qty(\va{r'})\va{r'}  ]\dd[3]{r'} = \qty[\frac{1}{2c} \int \va{r'} \cross \va{j}\qty(\va{r'}) \dd[3]{r'} ] \cross \frac{\va{r'}}{r^3} = \va{M} \cross \frac{\va{r}}{r^3}$$ 

If the loop is in plane,  $\int \va{r'} \cross \dd[3]{r'}$ is area and thus
$$\va{M} = \frac{I}{2c} \int \va{r'} \cross \dd[3]{r'} = \frac{I}{2c} \va{S}$$

\paragraph{Connection with angular momentum}
$$\va{j} = \rho_e \va{v}_e$$
Substituting
$$\va{M} = \frac{1}{2v} \int \va{r} \cross \rho_e \va{v}_e \dd[3]{r}$$
This looks pretty much like angular momentum
$$\va{L} =  \int \va{r} \cross \rho_m \va{v}_m \dd[3]{r}$$
If $\rho_m \propto \rho_e$ and $v_m \propto v_e$ (e.g., there is one kind of current carriers), $\va{L}\propto \va{M}$. For $v_e=v_m$, we get
$$\va{M} = \frac{q}{2mc} \va{L}$$
We define gyromagnetic relation $\frac{M}{L}$.
\paragraph{Bohr magneton} $$M_B = \frac{e\hbar}{2mc}$$
\paragraph{Electromagnetic force}
$$\va{F} = q\va{E}  + q\frac{\va{v}}{c}\cross \va{B}$$
If $\va{E} = 0$, power of force
$$\va{F} \vdot \va{B} = \qty(q\frac{\va{v}}{c}\cross \va{B}) \vdot \va{v}= 0$$

So for magnetic force:
$$\dd{\va{F}} = \frac{1}{c} \sum_{\alpha \in \dd{l}} q_\alpha \va{v}_\alpha \cross \va{B} = \frac{I}{c} \dd{\va{l}} \cross \va{B}$$

Suppose we have long wire with $\va{I} = I\vu{x}$ and external field $\va{B} = B\va{y}$. If ve move wire in direction $-\vu{z}$, we need to spend energy, however magnetic field doesn't apply work. Where the energy goes? Since now particles have velocity in direction $\vu{z}$, the field accelerates them.

\paragraph{Energy of $\va{B}$}
In case of electric field we got energy $\frac{E^2}{8\pi}$. Unsuprisingly for magnetic field we get $\frac{B^2}{8\pi}$.

Suppose we have infinite plane of current $\dd{I} = K \dd{l}$ and one parallel plane with current in opposite direction.

What is work per unit area required to move the plane by distance $h$?

Now since the field $B_0$ is constant inside and 0 outside, the work is exactly the energy of the magnetic field.

What is the field of the lower plane?
$$\curl{\va{B}} = \frac{4\pi}{c} \va{j}$$
$$\int \curl{\va{B}} \vdot \dd{\va{s}} = \oint \va{B} \vdot \dd{l}$$
$$\frac{4\pi}{c} KL_0 = 2BL_0$$
Thus
$$B = \frac{2\pi}{c} K$$

Since force is $\dd{\va{F}} =\frac{I}{c} \dd{\va{l}_I} \cross \va{B}$
where $\va{l}_I$ is direction of current. Then force on upper plane is
$$\dd{F} = \frac{I}{c}\dd{l_0}  = \frac{K}{c}\dd{l}\dd{l_0} B $$
And thus the work is
$$\dd{W} = \underbrace{h\dd{l}\dd{l_0}}_{\text{volume element}} \frac{K}{c}B = \dd{V} \frac{B^2}{2\pi} =  \dd{V} \frac{B^2_0}{8\pi}  $$
i.e., the energy per volume is $\frac{B^2}{8\pi}$.

\paragraph{Energy of magnetic dipole moment in external magnetic field}
