%TODO 
$$\int \va{r} $$
$$\int \va{r} \cross \frac{q}{c} \va{A}  \dd[3]{r}$$
$$P_{tot} = m\va{v} - \frac{q}{c} \va{A} $$

$$\dd{I}  =\frac{1}{4\pi c^3} \abs{\ddot{\va{d}} \cross \vu{r}}^2 \dd{\Omega}$$
Since
$$\expval{S} = \frac{c}{4\pi} \expval{B^2} \vu{r}$$
$$I = \int \dd{I} = \int_0^{2\pi} \dd{\phi} \int_0^\pi \dd{\theta} \frac{\ddot{\va{d}}^2}{4\pi c^3} \sin[3](\theta) \dd{\theta}\dd{\phi} = \frac{2}{3c^3}\abs{\ddot{\va{d}}}^2$$
\paragraph{Charge moving with constant acceleration}
$$\va{d} = q\va{r} \Rightarrow \ddot{\va{d}} = qa$$
i.e.,
$$I = \frac{2q^2a^2}{3c^3}$$

\paragraph{Dipole in circular motion}
$$\begin{cases}
d_x = d_0 \sin(\omega t)\\
d_y = d_0 \cos(\omega t)\\
\end{cases}$$
Then
$$\ddot{\va{d}}^2 = \omega_0^4 d_0^2$$
$$\expval{\dv{I}{\Omega}} = \frac{d_0^2\omega_0^4}{8\pi c^3} (1+\cos[2](\theta))$$
$$\va{I} = \frac{2q^2R^2\omega^4}{3c^3}$$
where $\theta$ is second spherical coordinate.
\paragraph{Example} How many time takes electron to fade in hydrogen atom as a result of energy loss.
\subparagraph{Solution}
$$P = \frac{2e^2R^2\omega t}{3c^3}$$
Suppose $e$ moves in circular trajectory:
$$\omega^2 = \frac{e^2}{mR^3}$$
since
$$\frac{mv^2}{R} = \frac{e^2}{R^2}$$
Initial energy is
$$E = \frac{1}{2}mv^2 - \frac{e^2}{R} = -\frac{1}{2} \frac{e^2}{R}$$
Since
$$P = -\dv{E}{t}$$
$$\frac{2e^6}{m^2R^4c^3} = \frac{e^2}{2R^2} \dv{R}{t}$$
We got $$\frac{E}{P} \sim \tau \sim 10^{10} s$$, i.e. hydrogen atom classically is impossible.

\subsection{Scattering}
Suppose we have wave coming to charged particle:
$$\va{E} = \va{E}_0 \cos(\omega t - \va{k} \vdot \va{r})$$
Writing Newton second law:
$$m\ddot{\va{r}} = q\va{E}_0 \cos(\omega t - \va{k} \vdot \va{r}) + q\frac{\va{v}}{c} \cross \va{B}_0\cos(\omega t - \va{k} \vdot \va{r}) $$
We neglect second-order fields, assuming $\frac{v}{c} \ll 1$. Also suppose$r(t=0) = 0$ and $\va{k} \vdot \va{r}(t) \ll 1$.
$$r(t) \ll \frac{1}{k} \sim \lambda$$
Since
$$\ddot{\va{d}} = q\ddot{\va{r}} = \frac{q^2}{m} \va{E}(t)$$
$$\dd{I} = \frac{q^4}{4\pi m^2 c^3} \qty(\va{E} \cross \vu{r})^2 \dd{\Omega}$$
i.e.,
$$\va{d} \propto \va{E}$$
and
$$\abs{\va{d}} \propto \abs{\omega^2 \va{E}}$$
Now
$$\dd{\sigma} = \frac{\dd{I}}{S}$$
,where
$$S = \frac{c}{4\pi}E^2$$
Differentiable cross section is
$$\dd{\sigma} = \qty(\frac{q^2}{mc^2} )^2\sin[2](\theta) \dd{\Omega}$$
$$\sigma = \frac{8\pi}{3}\qty(\frac{q^2}{mc^2} )^2\sin[2](\theta) $$