$$\va{\mu} = \va{M}(\va{r}) \Delta V$$
Then
$$\va{A}_n = \int \dd[3]{r'}\va{M}(\va{r}) \cross \frac{\va{r}-\va{r}'}{\abs{\va{r}-\va{r}'}^3} = \int \dd[3]{r'}\va{M}(\va{r}) \cross \grad_{r'} \frac{1}{\abs{\va{r}-\va{r}'}} = \oint \frac{\va{M}(\va{r}') \cross \dd{\va{r}'}}{\abs{\va{r}-\va{r}'}} + \int \frac{\grad_{r'}\cross \va{M}(\va{r}')}{\abs{\va{r}-\va{r}'}} \dd[3]{r'}$$
And total potential is
$$\va{A}_{tot} = \va{A}_{free} + \va{A}_{m} = \frac{1}{c} \int \dd[3]{r'} \frac{\va{j}(\va{r}') + c \curl{\va{M}(\va{r}')}}{\abs{\va{r}-\va{r}'}} + \frac{1}{c} \int \frac{\va{K}(\va{r}')\dd{s'} + c\va{M}(\va{r}') \cross \dd{\va{s}'}}{\abs{\va{r}-\va{r}'}}$$
We denote
$$\va{j}_b = c \curl{\va{M}}$$
and
$$\va{K}_b = c\va{M} \cross \vu{n}$$
\paragraph{Equations ofbmagnetic field in matter}
$$\curl{\va{B}} = \frac{4\pi}{c} \qty(\va{j} + c\curl{\va{M}})$$
Define
$$\va{H} = \va{B} - 4\pi \va{M}$$
And thus
$$\curl{\va{H}} = \frac{4\pi}{c} \va{j}$$
We define $\chi_M$ as
$$\va{M}(\va{r}) = \chi_m \va{H}$$
and
$$\va{B}= (1+4\pi \chi_M) \va{H} = \mu \va{H}$$
$\mu$ is called permeability.

\section{Time-dependent $\va{E}$ and $\va{B}$ }
\paragraph{Faraday's law of induction}
$\va{B}(t)$ induces current in closed circuit.
Define $\Phi$, flux of magnetic field
$$\Phi = \int \va{B} \vdot \dd{\va{s}}$$
then EMF, $\mathcal{E}$ is
$$\mathcal{E} = \oint \va{E} \vdot \dd{\va{r}} = -\frac{1}{c} \dv{\Phi}{t} =-\frac{1}{c} \dv{t} \int \pdv{\va{B}}{t} \vdot \dd{\va{s}}$$
\paragraph{Lenz's law}
The direction of current induced in a conductor by a changing magnetic field due to induction is such that it creates a magnetic field that opposes the change that produced it.
\paragraph{Different description of Faraday's law}
Suppose we have $\va{B}(\va{r})$ independent on time. For some loop of current
$$\va{F} = \frac{q}{c} \va{v} \cross \va{B}$$
Integral on force on charge over the loop
$$\oint \frac{q}{c} (\va{v} \cross \va{B}) \vdot \dd{\va{r}}$$
And thus
$$\va{E} = \frac{\va{v}}{c} \cross \va{B}_{lab}$$
Then EMF is
\begin{align*}
\mathcal{E} = \oint \frac{1}{c} (\va{v} \cross \va{B}) \vdot \dd{\va{r}} = \frac{1}{c} \int \curl(\va{v} \cross \va{B}) \vdot \dd{\va{s}} = \frac{1}{c} \int \qty[\va{v}\qty(\div{\va{B}}) - (v \vdot \grad)\va{B} ] \vdot \dd{\va{s}} =\\= -\frac{1}{c} \int \qty(\va{v} \vdot \grad) \va{B} \vdot \dd{\va{s}} = -\frac{1}{c} \int \dv{\va{B}}{t} \vdot \dd{\va{s}} = -\frac{1}{c} \dv{t} \int \va{B} \vdot \dd{\va{s}}
\end{align*}
\paragraph{Full derivative}
$$\dv{F(\va{r}, t)}{t} = \lim_{\Delta t \to 0} \frac{F(\va{r}\qty\big(t+ \Delta t), t+\Delta t) - F\qty\big(\va{r}(t), t)}{\Delta t} = \pdv{F}{t} + \qty(\va{v} \vdot \grad) F$$

\paragraph{Example}
$$\va{B}= \va{B}(\va{r}, t)$$
$$\int\limits_{S_0} \curl{\va{E}} \vdot \dd{\va{s}}=\oint \va{E} \vdot \dd{\va{r}} = -\frac{1}{c} \int\limits_{S_0} \pdv{\va{B}}{t} \vdot \dd{\va{s}}$$
Since this is independent on $S_0$, we get
$$ \curl{\va{E}} = -\frac{1}{c} \pdv{\va{B}}{t}$$
which is one of Maxwell's equations:
\paragraph{Maxwell's equations}
$$\begin{cases}
\div{\va{D}} = 4\pi \rho\\
\curl{E} = -\frac{1}{c} \pdv{\va{B}}{t} \\
\div{\va{B}} = 0\\
\curl{\va{H}} = \frac{4\pi}{c} \va{j} + \frac{1}{c}\pdv{\va{D}}{t}
\end{cases}$$
$ \frac{1}{c}\pdv{\va{D}}{t}$ is called displacement current, we can see that it's needed from $\div{\curl{H}} = 0$:
$$\pdv{\rho}{t} + \div{\va{j}} = 0$$

If there were magnetic monopoles we would have, in vacuum:
$$\begin{cases}
	\div{\va{E}} = 4\pi \rho\\
	\curl{E} = -\frac{1}{c} \pdv{\va{B}}{t} \textcolor{red}{-\frac{4\pi}{c}\va{j}_m} \\
	\div{\va{B}} = 0\textcolor{red}{+4\pi \rho_m}\\
	\curl{\va{B}} = \frac{4\pi}{c} \va{j} + \frac{1}{c}\pdv{\va{D}}{t}
\end{cases}$$
i.e., equations are fully symmetric.
\paragraph{Duality transformation}
If we define 
$$\begin{pmatrix}
\rho_e\\\rho_m
\end{pmatrix} = \begin{pmatrix}
\cos \psi&-\sin \psi\\\sin \psi&\cos \psi
\end{pmatrix}\begin{pmatrix}
\rho_e'\\\rho_m'
\end{pmatrix}$$
$$\begin{pmatrix}
\va{E}\\\va{B}
\end{pmatrix} = \begin{pmatrix}
\cos \psi&-\sin \psi\\\sin \psi&\cos \psi
\end{pmatrix}\begin{pmatrix}
\va{E}'\\\va{B}'
\end{pmatrix}$$
new quantities fulfill generalized Maxwell equations. If for all particles $\frac{q_m}{q_e} = \text{const}$, we can choose $\psi$ such that $\rho_m = 0$.

\subsection{Relativity}
\paragraph{Lorentz trasformation}
$$\begin{cases}
x = \frac{x'+v't}{\sqrt{1-\frac{v^2}{c^2}}}\\
t = \frac{t'+\frac{v}{c^2}x'}{\sqrt{1-\frac{v^2}{c^2}}}\\
y=y'\\
z=z'
\end{cases}$$
\paragraph{4-vector}
$x^0=ct$, $x^1=x$, $x^2=y$, $x^3=z$, this is contravariant 4-vector
$$x^i = \qty(ct, \va{r})$$
Covarient 4-vector is
$$x_i = \qty(ct,, -\va{r})$$
Einstein summation convention is
$$x_ix^i = \sum_{i=0}^3 = x_ix^i = (ct)^2 - r^2$$
\paragraph{Metric}
Define $g_{ij}$ such 
$$x_i = g_{ij} x_j$$
i.e.,
$$g = \begin{pmatrix}
1&0&0&0\\
0&-1&0&0\\
0&0&-1&0\\
0&0&0&-1
\end{pmatrix}$$
Denote $\eta=g$.
\paragraph{Examples of metrics}
In 3D, we have identity metric
$$\dd{r^2} = \dd{x^2} + \dd{y^2} + \dd{z^2}$$
However on sphere 
$$\dd{l^2} = \sin[2](\theta) \dd{\phi^2} +b\dd{\theta^2}$$
i.e. metric is not identity.
\paragraph{4-vector in electromagteics}
$$A^i = \qty(\phi, \va{A})$$
