
$$ \phi\qty(\va{r'}) = \underbrace{\int\limits_V    G \qty(\va{r}, \va{r'})\rho\qty(\va{r}) \dd[3]{\vb{r}}}_{0} - \oint\limits_S  \frac{\phi }{4\pi}\cdot \pdv{G}{\vu{n}} \dd{s} $$
$$\pdv{G}{\vu{n}} = -\eval{\pdv{G}{x}}_{x=0}$$
$$\phi = \frac{x'}{2\pi} \int \dd{z} \dd{y} \frac{\phi_S(y,z)}{\qty[x'^2 + (y-y')^2 + (z-z')^2]^{\frac{3}{2}}}$$
\paragraph{Symmetry of Green function}
$$G \qty(\va{r}, \va{r'}) = G \qty(\va{r'}, \va{r})$$

\subsection{Separation of variables}

Suppose we have two planes parallel to $y$ axis with zero potential with distance $L$ between them, and a plane parallel to $x$ axis with potential $V(x)$. We want to solve
$$\laplacian{ \phi} = 0$$
$$\pdv[2]{\phi}{x}+\pdv[2]{\phi}{y}+\pdv[2]{\phi}{z}=0$$
Since there is no change on $z$ direction we get
$$\pdv[2]{\phi}{x}+\pdv[2]{\phi}{y}=0$$
Let's use anzatz for the solution $\phi = X(x)Y(y)$:
$$Y\dv[2]{X}{x} + X\dv[2]{Y}{y} = 0$$
$$\frac{1}{X}\dv[2]{X}{x} + \frac{1}{Y}\dv[2]{Y}{y} = 0$$
Since we have operands depending on different variables, both are constant.
$$\begin{cases}
\frac{1}{Y}\dv[2]{Y}{y} = +k^2\\
\frac{1}{X}\dv[2]{X}{X} = -k^2\\
\end{cases}$$
(since we want $Y$ to decrease to $0$ in infinity, we choose positive value for $Y$ and negative for $X$).

The solution is
$$\begin{cases}
X = a \cdot \sin(kx) + b \cdot \cos(kx)\\
Y = c \cdot e^{ky} + d \cdot e^{-ky}
\end{cases}$$
Since $\phi(0,y,z) = \phi(L,y,z) = 0$, $b=0$ and $k$ acquires discrete values $k_n = \frac{n\pi }{L}$.
Since $\phi(x, y\to \infty, z) = 0$, $c=0$.

Now we need to force $\phi(x,y=0,z)=V(x)$:
$$\Phi = X(x)Y(y=0) = V(x)$$

We know that the solution is of form
$$d_na_n \sin (k_nx) e^{-k_ny}$$
Thus we can get a general solution in a form
$$\phi = \sum d_na_n \sin(k_n x) e^{-k_ny}$$
So we want
$$V(x) = \sum \underbrace{d_na_n}_{q_n} \sin(k_nx)$$
which is Fourier series:
$$\int \dd{x} V(x) \sin(k_{n'}x) = \sum_{n} \int \dd{x} q_n \sin(k_nx) \sin(k_{n'}x) = \sum_n q_n \frac{L}{2} \delta_{nn'} = \frac{Lq_{n'}}{2} $$

\paragraph{Example}
Suppose we have to planes with angle $\theta_0$ between them and potential $\phi_0$ on both. In cylindrical coordinates
$$\laplacian{\phi} = \frac{1}{R} \pdv{R} \qty(R\pdv{\phi}{R}) + \frac{1}{R^2} \pdv[2]{\phi}{\theta} + \pdv[2]{\phi}{z} $$
By separation of variables
$$\phi = F(R)G(\theta)$$
$$\frac{G(\theta)}{R} \dv{F}{R} + G\dv[2]{F}{R} + \frac{F(R)}{R^2} \dv[2]{G}{\theta} = 0$$
$$\begin{cases}
\frac{1}{G} \dv[2]{G}{\theta} =-\nu^2 \\
\frac{1}{F} \dv{F}{R} + \frac{R^2}{F}\dv[2]{F}{R} = \nu^2
\end{cases}$$
\subparagraph{$\nu>0$}
$$\begin{cases}
G = A\cos(\nu \theta) + B\sin(\nu \theta)\\
F = aR^{-\nu} + bR^{\nu}
\end{cases}$$
\subparagraph{$\nu=0$}
$$\begin{cases}
G=\tilde{A}+\tilde{B}\theta\\
F=\tilde{a}+\tilde{b}\ln R
\end{cases}$$
We get $\tilde{b}=0$ such that $F$ doesn't diverge in 0. Also there shouldn't be dependence on angle, so $\tilde{B}=0$ and $\tilde{A} \neq 0$, thus for $\nu=0$ potential is constant.

For positive $\nu$, $A=0$ and $a=0$ and also want $\sin(\nu \theta_0) = 0$ thus
$$\phi(\theta) = \phi_0 + \sum_{n=1} a_n R^{\frac{n\pi}{\theta_0}} \sin(n\pi \frac{\theta}{\theta_0}) $$

If $R \to 0$, the most dominant element of sum is $n=1$. Thus
$$\phi \propto R^{\frac{\pi}{\theta_0}}$$
i.e., due to $E \sim -\pdv{\phi}{R} \sim R^{\frac{\pi}{\theta_0} - 1}$:
$$\begin{cases}
E \to 0 & \theta_0 < \pi \\
E \to \infty & \theta_0 > \pi \\
\end{cases}$$